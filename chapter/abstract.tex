\begin{abstract}
For autonomous robot navigation, vision-based system can be easily deployed with 
inexpensive commercial cameras, especially for monocular visual navigation system. 
However, visual sensors are sensitive to environmental changes caused by lighting, object movement, weathers and seasons.
For both indoor and outdoor autnomous navigation, the changes over time can lead to poor localization
and deterioration in map quality. This may lead to a significant drop on robustness, especially for tasks involving repeated traversals such as autnomous path following, i.e. visual teach and repeat task.
To enable long-term autonomous navigation system, the robustness of the system is a must.
While extensive techniques are proposed to strengthen the robustness of localization
through image retrieval or deep-trained image features that are invariant to appearance changes and viewing angle.
They demonstrate impressive performance on changes caused by lighting condition variantion and seasonal changes, 
while they rarely focus on the changes caused by object movement. This kind of changes are rather typical
in situations such as parking lots, warehouses and offices. The movement can cause not only the appearance change,
but also the infeasibility of the previous paths. In extreme condition, object movement may lead to the
number of features to be insufficient, thus the localization fails.
Besides, localization and mapping are tightly-coupled, however, current researchs lack of an emphasis 
on map managment and update for long-term navigation tasks.
Therefore, this proposal focus on the long-term navigation system enabled by visual sensors and targets on
test and improve the localization robustness under environmental changes aroused by object movements.
Moreover, the corresponding map maintance algorithms aiming at record and update the surrounding changes is 
proposed as an important research direction. With the robustness improvement on localization and
mapping under changing environments, the long-term autnomous navigation tasks e.g. autonomous route following,
should be achieved.
\end{abstract}
