\begin{abstract}
For autonomous robot navigation, a vision-based system can be easily deployed with 
inexpensive commercial cameras, especially for a monocular visual navigation system. 
However, visual sensors are sensitive to environmental changes caused by lighting, object movement, weather, and seasons.
For both indoor and outdoor autonomous navigation, the changes over time can lead to poor localization
and deterioration in map quality. This may lead to a significant drop in robustness, especially for tasks involving repeated traversals such as autonomous path following, i.e. visual teach-and-repeat task.
To enable a long-term autonomous navigation system, the robustness of the system is a must.
While extensive techniques are proposed to strengthen the robustness of localization
through image retrieval or deep-trained image features that are invariant to appearance changes and viewing angle.
They demonstrate impressive performance on changes caused by lighting condition variation and seasonal changes, 
while they rarely focus on the changes caused by object movement. These kinds of changes are rather typical in situations such as parking lots, warehouses, and offices. The movement can cause not only the appearance change but also the infeasibility of the previous paths. In extreme conditions, object movement may lead to the
number of features being insufficient, thus the localization fails.
Besides, localization and mapping are tightly-coupled, however, current research lacks an emphasis on map management and update for long-term navigation tasks.
Therefore, this proposal focuses on the long-term navigation system enabled by visual sensors and targets to test and improve localization robustness under environmental changes aroused by object movements. Moreover, the corresponding map maintenance algorithms aiming at recording and updating the surrounding changes are proposed as an important research direction. With the robustness improvement on localization and mapping under changing environments, the long-term autonomous navigation tasks e.g. autonomous route following should be achieved.
\end{abstract}
