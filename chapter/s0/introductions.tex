\section{Background}
Autonomous mobile robotic systems are emerging dramatically in recent years from
unmanned aerial vehicle (UAV) to unmanned ground vehicle (UGV), from self-driving
vehicle to indoor service robots. A large portion of industrial applications have
already achieved high-level autonomy by current state of the art techniques. 
For industrial robots, the operation environment is usually structural constrained 
and fixed with various of customized assiting module such as using pre-installed 
external marker system to assist localization. 
Therefore, the industrial environment is barely changed with high stability, 
while for other demanding scopes of applications (e.g. autonomous driving, warehouse robots,
and service robots), the operational surrounding changes significantly with high dynamics
and uncertainties. This highlights the key focus on robustness and reliability under
changing environments for the wide spread of autonomous mobile robots. 

Comparing to 3D laser scanner, also know as Light detection and ranging (LiDAR), 
visual sensors (mostly referred to cameras) are cheaper and able to provide rich
texture information, which is more suitable for high level interactive tasks such
as mobile manipulation, human robot interaction. Cameras are also light-weight which enables massive deployment on all kinds of robotic platform, e.g. on micro aerial 
robotics, LiDARs are not suitable for its weight and size, as well as high drive power source requriement. Therefore, vision-based localization and mapping techniques are 
important which can achieve comparable performance with much lower hardware requriements.

This proposal focus on the vision-based long-term navigtaion tasks, which requires 
high robustness. During the task completion, the robot might encounter environment with uncertainty, such as highly dynamic objects (human, vehicles and other operating robots), moderately dynamic objects (boxes, bins and chairs), static objects but could be changed or moved (fences, wall and trees). This uncertain of surrounding changes 
can lead to the failure of localization and inaccuracy of the map. Without doubt, localization of robot is the fundermental problem which other operations relies on.
Regarding the map accuracy, the incremental changes and possible loss of localization, makes it important to maintaince an up-to-date map. Moreover, the discovery of human's memory system shows the mechanism behind an incrementally and continuously learning system. The human memory system contains temporal memory as well as long-term memory. Current research seldom tackles this problem while mostly focus on the temporal navigation and assume the environment to be static. 
\cite{yin2022bioslam}

for human perception 
of a scene, there might be multiple layer of memory. For example, people tends to remeber the scene appearance with 


Visual sensors, normally referred to monocular camera, stereo camera and RGB-D camera. 

Sensor, vision  

current research focus, deep learning, vo 



\section{Related Work}

