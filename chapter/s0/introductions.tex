\section{Background}
Autonomous mobile robotic systems are emerging dramatically in recent years from
unmanned aerial vehicle (UAV) to unmanned ground vehicle (UGV), from self-driving
vehicle to indoor service robots. A large portion of industrial applications have
already achieved high-level autonomy by current state of the art techniques. 
For industrial robots, the operation environment is usually structural constrained 
and fixed with various of customized assiting module such as using pre-installed 
external marker system to assist localization. 
Therefore, the industrial environment is barely changed with high stability, 
while for other scopes of applications (e.g. autonomous driving, warehouse robots,
and service robots), the operational surrounding changes significantly with high dynamics
and uncertainties. This highlights the key focus on robustness and reliability under
changing environments for the spread of autonomous mobile robots. 

A mobile robot navigation system consists of a perception module, a path planning module, and a control module.
A perception module is responsible for scene understanding which may contain localization, mapping as well as 
object detection. For a perception module, the sensor can be visual sensors, 


Visual sensors, normally referred to monocular camera, stereo camera and RGB-D camera. 

Sensor, vision  

current research focus, deep learning, vo 



\section{Related Work}



Intelligent robots are influencing many aspects of the world nowadays, from collaborative robot arms in factories to L4 autonomous driving technology, from biped household robots to quadruped mobile military agents, and from unmanned surface vehicles to quadrotor swarms. The more deeply we imagine the future, the more indispensable we find robot services.

For mobile robots, navigation is always the kernel function. However, compared with the automation of manipulation, mobile robot navigation is evolving more slowly. In the classical manipulation and manufacturing scenarios, the workspace of robots is usually well defined, and the robots can perform correctly under human-designed programming, without any interaction between uncertain objects and themselves. If we regard \textit{replacing the repetitive workloads} as the first step, we should start to consider \textit{living with the robot} in the next step. For mobile robots, to be incorporated into human's daily life, they must be intelligent in unknown and pedestrian-rich environments for tasks like autonomous driving, cargo delivers and household mobile robots, which is \textit{behaving like a human}.

Human beings can navigate in crowded environments smartly without dependence on pre-defined high-resolution maps. We can also explore unknown environments without taking time to think about the information gain or frontiers. To navigate safely and efficiently like a human being, mobile robots should be able to perceive and predict behaviours of unfamiliar and dynamic agents. Based on the implicit or explicit understanding, an accessible policy considering various constraints is needed to guide the robots.


Deep learning, as a solution for artificial intelligence, is capable of building progressively meaningful feature abstraction of input data.
It plays an essential role in various fields of study \cite{Goodfellow-et-al-2016}, bringing the state of the art in image classification \cite{he2016deep,huang2017densely, krizhevsky2012imagenet},
semantic segmentation \cite{chen2016deeplab, long2015fully},
human-level game playing \cite{mnih2016asynchronous, mnih2015human}, driving real robotic systems in navigation \cite{tai2017virtual, zhang2017deep, zhu2017target}
and manipulation \cite{levine2016end,yu2018one} tasks.

We may be witnessing the most rapidly growing trend of deep learning techniques for robotics tasks in recent years.
Replacing hand-crafted features with learned hierarchical distributed deep features, and learning control policies directly from high-dimensional sensory inputs, the robotics community is making substantial progress towards building fully autonomous intelligent systems.
